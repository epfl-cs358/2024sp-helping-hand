\section{Risk Assessment}

In this section, we conduct a comprehensive risk assessment to identify potential challenges and vulnerabilities that could impede the success of our project.
By evaluating technological risks we aim to proactively address these issues through strategic planning and prototype development.
This assessment underscores our commitment to ensuring the resilience and adaptability of our project, thereby minimizing the likelihood of failure and maximizing its overall effectiveness.

\begin{itemize}
    \item Power supplies to servos:\\
        There is a risk associated with supplying adequate and stable power to the servos from the ESP32.
        Inadequate power supply can lead to erratic behavior or malfunctioning of the servos, potentially putting at risk the functionality of the project.
    \item Adaptability to different remotes:\\
        The project's adaptability to different remote controls is a challenge, as compatibility issues may arise with various types of remotes.
    \item Magnet interferences:\\
        The presence of magnets in the project could disrupt the operation of sensitive electronic components such as motors or the WiFi chip of the ESP32, leading to unpredictable behavior.
    \item WiFi network stability:\\
        The reliance on WiFi connectivity on those ESP32 microcontrollers introduces a risk related to network instability, which could affect communication between project components.
    \item Ease of configuration:\\
        The complexity of configuring and calibrating the project components may pose a risk, particularly in terms of user-friendliness and accessibility for non-technical users. 
\end{itemize}
