\section{Risk Assessment}

In this section, we conduct a comprehensive risk assessment to identify potential challenges and vulnerabilities that could impede the success of our project.
By evaluating technological risks we aim to proactively address these issues through strategic planning and prototype development.
This assessment underscores our commitment to ensure the resilience and adaptability of our project, thereby minimizing the likelihood of failure and maximizing its overall effectiveness.

% reword the below p
%  remote adaptability: required pressure for each remote is different: some hard-pressed, some don't require as much pressure to acknowledge a button press. another problem is that the remote sizes
%                         vary a lot in length but especially in width, which makes designing and creating a clamp that fixes most of the remotes difficult.  
%  calibration: the calibration as a first step is to be done manually, which would include manually moving the pushing mechanism to the button via the phone application and setting it up as a button. This might be somewhat 
%                 time consuming and hard to configure for non-technical users. 
%                 The next step to the setup includes computer vision.
%                 This also comes with its own challenges, and is complex because of the camera quality of the ESP32-CAM, the nature of remotes and the setup restrictions: small and non-tv remotes are likely to have similar colored buttons to 
%                 the remote's main body, and this makes recognition of individual buttons difficult. Another issue is that the setup, i.e. placing the remote in the clamp is done by a person, and this causes reliability issues
%                 in the sense that not everyone can place and clamp the remote perfectly, and this will cause a sligtly different placement each time. This will cause in a different lightning and a different angle of view for
%                 the camera to pick up. This might affect the accuracy of the image recognition. The computer vision part requires an external computer to set up button positions, and this information is then
%                 sent to the phone application, Here, the communication between the computer, the phone app and the plotter is crucial and complicated. 

\begin{itemize}
    \item Power supplies to steppers:\\
        As mentioned in the manual, inadequate wiring practices and carelessness may lead to frying a computer or damaging the parts, especially the motor drivers and the stepper motors, potentially putting at risk the functionality of the project.
    \item Adaptability to different remotes:\\
        The project's adaptability to different remote controls is a challenge, as compatibility issues may arise with various types of remotes due to the vast spectrum of sizes. The amount of different size and shapes possible makes it complicated to design the clamp mechanism. 
    \item Magnet interferences:\\
        The presence of magnets in the project could disrupt the operation of sensitive electronic components such as the solenoid push-down mechanism, motors or the WiFi chip of the ESP32, leading to unpredictable behavior.
    \item WiFi network stability:\\
        The reliance on WiFi connectivity on those ESP32 microcontrollers introduces a risk related to network instability, which could affect communication between project components.
    \item Ease of configuration:\\
        The configuration and calibration of the plotter will have the possibility to be done in two ways: manually via the phone app or with computer vision via a computer. The manual setup may be time consuming for remotes with many buttons. 
        The camera and computer vision setup may be hard to configure for non-technical users. This option also has its own additional challenges. The recogni̇ti̇on of the remote and its buttons rely on the camera, the placement of the remote, the lightning of the room
        and overall restrictions due to remote design. These may affect the accuracy of the image recognition.
    \item Computer vision to button press precision:\\
        Detecting the correct position and therefore being able to guarantee a precise button press while using computer vision to calibrate and configure the device could be difficult for the same reasons listed in above point.
\end{itemize}
